\documentclass[../Main.tex]{subfiles}
\begin{document}
The background section of this thesis provides a comprehensive overview of the key concepts and technologies that serve as the foundation for our proposed solution. This chapter examines the fundamental characteristics of blockchain technology, what social login for Web3 is, what a smart contract is, and the fundamental comprehension and utilization scenarios of Shamir's secret sharing and distributed key generation. Understanding these concepts is crucial for appreciating our solution's motivations and its potential impact on the decentralized digital landscape.
\section{Blockchain}
\label{section:2.1}
Blockchain technology has emerged as a revolutionary innovation with the potential to transform industries and revolutionize how digital transactions are conducted. It provides a decentralized and transparent platform for secure and unchangeable record-keeping, eliminating the need for intermediaries and facilitating peer-to-peer interactions. This chapter provides a concise introduction to blockchain technology, highlighting its historical context, the problem it seeks to solve, and the primary contributions of its first author. In 2008, an anonymous person or group of people using the alias Satoshi Nakamoto \cite{Bitcoin} introduced the concept of blockchain for the first time. The seminal whitepaper titled "Bitcoin: A Peer-to-Peer Electronic Cash System" by Satoshi Nakamoto outlined the fundamental principles and architecture of blockchain technology as a solution to the issues of trust and decentralized digital currency. Bitcoin's introduction of blockchain represented a significant milestone in the evolution of cryptocurrencies and decentralized systems. Traditional centralized systems' lack of trust and security is the issue blockchain seeks to address. The reliance of centralized systems on a single trusted authority to validate and authenticate transactions leaves room for manipulation, deception, and censorship. Blockchain technology addresses these issues by establishing a decentralized network of nodes where consensus mechanisms guarantee the validity and integrity of transactions without requiring a central authority. The first author, Satoshi Nakamoto, introduced a secure and decentralized framework for digital currency transactions, laying the groundwork for blockchain technology. Combining existing cryptographic techniques, such as hash functions and digital signatures, with a distributed ledger system was Nakamoto's most significant innovation. This innovation facilitated the creation of a transparent and tamper-resistant ledger of transactions, ensuring the integrity and immutability of blockchain data. Since Nakamoto's original work, blockchain technology has expanded beyond cryptocurrencies such as Bitcoin. It has implications in numerous industries, including finance, supply chain management, and healthcare, among others. The blockchain's decentralized nature provides opportunities for greater transparency, efficiency, and trust in these industries, paving the way for innovative solutions and new business models.



\subfile{background/blockchain.tex}
\section{Shamir's secret sharing}
Shamir's secret sharing is a fundamental cryptographic algorithm that plays a vital role in assuring the secure distribution and storage of data across a variety of applications. This ingenious algorithm, developed by Adi Shamir in 1979, employs the principles of polynomial interpolation to divide a confidential secret into multiple shares. Then, these shares are distributed to various participants, each of whom holds a unique piece of the puzzle. The genius resides in the fact that the original secret can only be reconstructed by combining a minimum number of shares. The implementation of Shamir's secret sharing begins with the selection of the secret to be protected. The secret is then used as the constant element in a polynomial of a specified degree. This polynomial is the basis for constructing the shares. The evaluation of the polynomial at specific points corresponds to the allocation of shares to each participant. The flexibility of Shamir's secret sharing is its greatest asset. The threshold required to reconstruct the secret can be modified based on the system's particular requirements. If the threshold is set to "k," for instance, any combination of "k" or more shares can be used to recover the original secret. As long as the minimum threshold is maintained, this provides a robust mechanism for ensuring resilience against loss or larceny of shares. Shamir's disclosure of a secret has a wide range of applications in various disciplines. It is frequently employed in cryptographic key management, in which a sensitive cryptographic key is divided into portions and distributed to key holders. This adds an additional layer of security to critical systems by ensuring that no single individual can access the key without the cooperation of multiple parties.//
\indent Here is a straightforward illustration of how Shamir's secret sharing works.
Suppose we wish to divide a secret value of 42 into 5 portions with a threshold of 3. By evaluating a polynomial at various points, the shares are generated.
Share 1: (1, 17), Share 2: (2, 23), Share 3: (3, 38), Share 4: (4, 14)
Share 5: (5, 7)
x represents the point on the polynomial curve, while y is the value of the polynomial at that point.
Now, we need at least three shares to reconstruct the secret. Consider the shares 2, 3, and 4. We can use these shares to interpolate the polynomial and determine the value at x = 0 that corresponds to our confidential value of 42 by employing interpolation.
Using the Lagrange interpolation formula \cite{larrange-interpolation}, the secret can be calculated:
\[
\text{Secret} = \frac{{23 \cdot (0 - 3) \cdot (0 - 4)}}{{(2 - 3) \cdot (2 - 4)}} +
\frac{{38 \cdot (0 - 2) \cdot (0 - 4)}}{{(3 - 2) \cdot (3 - 4)}} +
\frac{{14 \cdot (0 - 2) \cdot (0 - 3)}}{{(4 - 2) \cdot (4 - 3)}}
\]
After simplifying the equation, the hidden value is determined to be 42.

\section{Distributed Key Generation}
The Distributed Key Generation (DKG) protocol is at the vanguard of contemporary cryptographic mechanisms, offering a revolutionary method for collaboratively generating shared secret keys among multiple parties without the need for a centralized trusted authority. By removing the need for a single point of control, the DKG protocol significantly improves security, making it an indispensable instrument for protecting sensitive data and vital systems. The fundamental principle of the DKG protocol is its capacity to decentralize the generation and distribution of the secret key among all participants. This collaborative effort ensures that no single party has complete knowledge of the confidential key, thereby mitigating the risk of a single point of failure and reducing the number of exploitable vulnerabilities. The protocol consists of distinct segments, each of which plays a vital role in generating a shared secret key, the protocol's ultimate objective. In the initial stages of the key generation phase, the parties generate the secret key jointly. The distribution of these shares to the participants signifies the distribution phase. Verification is the next stage, during which each party verifies the received shares to ensure their authenticity and integrity. The true strength of the DKG protocol rests in the reconstruction phase, where the mathematical combination of the distributed shares is used to reconstruct the final confidential key. Importantly, the success of this reconstruction requires the participation of a minimum number of trustworthy parties. This threshold ensures that even if some parties are compromised or conduct maliciously, the security of the final secret key remains uncompromised so long as the minimum number of honest participants is maintained. Numerous advantages make DKG protocols an adaptable and indispensable resource for a variety of applications. DKG protocols enable parties to collaboratively perform computations on sensitive data without disclosing their individual inputs, ensuring privacy and confidentiality. The DKG protocol facilitates the secure generation and distribution of cryptographic keys in cryptographic key management, safeguarding against unauthorized access and key compromise. Moreover, DKG protocols benefit threshold cryptography by enabling secure operations that necessitate the participation of a predefined threshold of parties. Secure communication protocols can also utilize DKG protocols to establish secure channels and prevent data interception and surveillance.
\indent The Pedersen DKG protocol, proposed by Torben Pedersen \cite{pedersen1991threshold} in 1991, was used in the thesis. The Pedersen DKG protocol utilizes polynomial interpolation techniques and cryptographic primitives to achieve secure and distributed key generation. It provides a robust mechanism for establishing shared secret keys without relying on a trusted central authority. The protocol involves several steps, including key generation, sharing, verification, and reconstruction.The Pedersen DKG protocol utilizes polynomial interpolation techniques and cryptographic primitives to achieve secure and distributed key generation. It provides a robust mechanism for establishing shared secret keys without relying on a trusted central authority. The protocol involves several steps, including key generation, sharing, verification, and reconstruction.
\section{Smart contract}
\subfile{background/smartcontract.tex}

\section{Executor for DApps}
In the domain of blockchain-based decentralized applications (DApps), the executor plays a crucial role in facilitating the execution of blockchain network transactions. As the backend component of a DApp, the executor functions as an intermediary between the user-facing frontend and the blockchain. It is primarily responsible for establishing a connection to the blockchain, submitting transactions on behalf of users, and managing various interactions with the smart contract. Management of the user's private key or mnemonic is a fundamental aspect of an executor's functionality. A crucial piece of cryptographic information, the private key enables the authentication of transactions and verifies the identity of blockchain users. In some implementations, the private key is stored as a mnemonic, which is a string of phrases that serves as a human-readable representation of the key. By storing the mnemonic securely in the backend, the executor can access it when required to sign transactions on behalf of the frontend, thereby ensuring a seamless and secure user experience. Executors serve as verifiers and assigners within the Perdesen Distributed Key Generation (DKG) protocol in the context of the thesis. As part of this cryptographic protocol, multiple participants generate a shared secret key without relying on a central authority that can be trusted. The executors are responsible for confirming the correctness and impartiality of the smart contract's round assignment procedure. Executors contribute to the overall security and resilience of the decentralized system by functioning as dependable parties. Their function in the DKG protocol ensures that the key generation and distribution process is carried out with diligence and impartiality, thereby protecting the shared secret key's integrity. In addition, their participation in validating the veracity of the protocol adds an additional layer of confidence to the entire system, as it ensures that the generated secret key is genuine and can be used securely in cryptographic operations.

\section{Social login for Web3}
\subsection{Web3 and Dapps}
Web3 and Decentralized Applications (DApps) have emerged as critical components of the evolution of the internet and digital ecosystems. Web3 is the vision of a more decentralized and user-centric internet, in which individuals have greater control over their data, identity, and digital interactions. It is a collection of technologies, protocols, and frameworks designed to empower users, cultivate trust, and facilitate peer-to-peer interactions. DApps, on the other hand, are applications that are created on top of decentralized networks and typically utilize blockchain technology. These applications inherit Web3's fundamental principles, including decentralization, transparency, and user ownership. From financial services and governance platforms to gaming and social media applications, they provide a variety of features. 
The development and adoption of Web3 and DApps have been supported by a growing body of research and innovation. Several academic papers and technical publications have contributed to the advancement of these technologies. For instance, the paper by Wood et al. titled "Ethereum: A Secure Decentralized Generalized Transaction Ledger" provides a comprehensive overview of the Ethereum platform and its underlying principles \cite{Ethereum}.
Another significant contribution is the work by Swan, who explores the concept of "Token Economy" in his book "Token Economy: How the Web3 Reinvents Value Exchange." The book delves into the transformative potential of tokenization and its implications for various industries \cite{swan2018token}. Moreover, the paper by Buterin et al. titled "A Next-Generation Smart Contract and Decentralized Application Platform" introduces the Ethereum platform, highlighting its unique features and use cases \cite{buterin2014next}. This paper serves as a foundational reference for understanding the capabilities and potential of DApps built on Ethereum.
\subsection{Social login and OAuth}
Social login is a prevalent method of authentication that enables users to log in to websites and applications using their existing social media accounts. Users are no longer required to establish new accounts and remember additional login credentials. Instead, users can merely click on a social media button, such as "Sign in with Facebook" or "Sign in with Google," to authenticate themselves. Social registration is supported by the OAuth2 (Open Authorization 2.0) \cite{oauth2} protocol, which provides a secure and standardized authentication framework. When a user logs in with a social media account, the website or application sends them to the respective social media platform for authentication. The user is then presented with a consent interface that describes the data to which the website or application requests access. Once the user grants permission, the social media platform provides the website or application with an access token that can be used to retrieve user information and authenticate the user's identity. Using OAuth2 for social authentication has multiple advantages. It increases security by removing the need for websites and applications to store user credentials. Instead, the obligation for authentication falls on the shoulders of the most reputable social media platforms. Second, social login streamlines the user experience by allowing users to log in with a few clicks and avoid the inconvenience of creating new accounts. It also allows websites and applications to utilize the extensive user profile data available on social media platforms, including user names, profile pictures, and email addresses, for personalization and customization.
\subsection{Social login benefits DApps}
The technical complexity of Web3 technology, with its underlying blockchain, cryptographic keys, and smart contracts, can be intimidating for ordinary consumers. Understanding and navigating these complexities can be a significant barrier to entry for non-technical individuals, impeding the widespread adoption and utility of Web3 platforms. The decentralized nature of Web3 can exacerbate this problem, resulting in fragmented user experiences and inconsistent user interfaces, which makes it difficult for non-technical users to interact effectively with decentralized applications. To resolve these issues, it is essential to simplify the user experience and increase Web3 technologies' accessibility. By refining complicated processes and providing user-friendly interfaces, we can enable regular users to exploit Web3's potential without being overwhelmed by its technicalities. A design that is user-friendly and intuitive can make Web3 applications more accessible to the general public, thereby promoting their widespread adoption. Using smart contracts, the Pedersen Distributed Key Generation (DKG) protocol, and Shamir secret sharing, the proposed solution presented in this thesis seeks to enhance the usability of Web3 applications. Using these cryptographic techniques, the system can generate keys in a secure and distributed manner, thereby protecting sensitive data and enhancing user privacy and security. In addition, the solution addresses the difficulties encountered by non-technical users by removing the complexities of Web3 technologies, allowing them to interact with the system without difficulty. By incorporating smart contracts, users can access decentralized applications without needing to delve into the fundamentals of blockchain operations. The Pedersen DKG protocol and Shamir secret sharing mechanisms provide a robust and secure method for generating and exchanging cryptographic keys, ensuring that users retain control over their data while still taking advantage of Web3's decentralized nature. This thesis ultimately seeks to democratize Web3 technologies by making them more inclusive and accessible to a wider audience. By lowering entry barriers and enhancing the user experience, the proposed solution aims to unleash the full potential of Web3 for regular users, allowing them to confidently and easily participate in the decentralized ecosystem. The objective is to advance a more user-friendly, secure, and decentralized digital landscape through a combination of technological innovations and user-centric design.








%%%%%%%%%%%%%%%%%%%%%%%%%%%%%%%%%%%

\end{document}
