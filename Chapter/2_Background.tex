\documentclass[../Main.tex]{subfiles}
\begin{document}
The background section of this thesis provides a comprehensive overview of the key concepts and technologies that serve as the foundation for our proposed solution. This chapter examines the fundamental characteristics of blockchain technology, what social login for Web3 is, what a smart contract is, and the fundamental comprehension and utilization scenarios of Shamir's secret sharing and distributed key generation. Understanding these concepts is crucial for appreciating our solution's motivations and its potential impact on the decentralized digital landscape.
\section{Blockchain}
\label{section:2.1}
Blockchain technology has emerged as a revolutionary innovation with the potential to transform industries and revolutionize how digital transactions are conducted. It provides a decentralized and transparent platform for secure and unchangeable record-keeping, eliminating the need for intermediaries and facilitating peer-to-peer interactions. This chapter provides a concise introduction to blockchain technology, highlighting its historical context, the problem it seeks to solve, and the primary contributions of its first author. In 2008, an anonymous person or group of people using the alias Satoshi Nakamoto \cite{Bitcoin} introduced the concept of blockchain for the first time. The seminal whitepaper titled "Bitcoin: A Peer-to-Peer Electronic Cash System" by Satoshi Nakamoto outlined the fundamental principles and architecture of blockchain technology as a solution to the issues of trust and decentralized digital currency. Bitcoin's introduction of blockchain represented a significant milestone in the evolution of cryptocurrencies and decentralized systems. Traditional centralized systems' lack of trust and security is the issue blockchain seeks to address. The reliance of centralized systems on a single trusted authority to validate and authenticate transactions leaves room for manipulation, deception, and censorship. Blockchain technology addresses these issues by establishing a decentralized network of nodes where consensus mechanisms guarantee the validity and integrity of transactions without requiring a central authority. The first author, Satoshi Nakamoto, introduced a secure and decentralized framework for digital currency transactions, laying the groundwork for blockchain technology. Combining existing cryptographic techniques, such as hash functions and digital signatures, with a distributed ledger system was Nakamoto's most significant innovation. This innovation facilitated the creation of a transparent and tamper-resistant ledger of transactions, ensuring the integrity and immutability of blockchain data. Since Nakamoto's original work, blockchain technology has expanded beyond cryptocurrencies such as Bitcoin. It has implications in numerous industries, including finance, supply chain management, and healthcare, among others. The blockchain's decentralized nature provides opportunities for greater transparency, efficiency, and trust in these industries, paving the way for innovative solutions and new business models.



\subfile{background/blockchain.tex}

\section{Shamir's secret sharing}
Shamir's secret sharing is a cryptographic algorithm that divides a secret into multiple shares, which can then be distributed to various participants. The secret can only be reconstructed by combining a substantial number of shares. The algorithm was created in 1979 by Adi Shamir and is based on polynomial interpolation. To implement Shamir's secret sharing, a secret is first selected, and then a polynomial of a specific degree is generated with the secret as the constant term. The polynomial is subsequently evaluated at particular points to generate the shares. Each participant receives a portion of the polynomial curve that corresponds to a specific point. Any subset of shares, so long as it satisfies a certain threshold, can be used to reconstruct the original secret, according to Shamir's secret sharing scheme.\\
Here is a straightforward illustration of how Shamir's secret sharing works.
Suppose we wish to divide a secret value of 42 into 5 portions with a threshold of 3. By evaluating a polynomial at various points, the shares are generated.
Share 1: (1, 17), Share 2: (2, 23), Share 3: (3, 38), Share 4: (4, 14)
Share 5: (5, 7)
x represents the point on the polynomial curve, while y is the value of the polynomial at that point.
Now, we need at least three shares to reconstruct the secret. Consider the shares 2, 3, and 4. We can use these shares to interpolate the polynomial and determine the value at x = 0 that corresponds to our confidential value of 42 by employing interpolation.
Using the Lagrange interpolation formula \cite{larrange-interpolation}, the secret can be calculated:
\[
\text{Secret} = \frac{{23 \cdot (0 - 3) \cdot (0 - 4)}}{{(2 - 3) \cdot (2 - 4)}} +
\frac{{38 \cdot (0 - 2) \cdot (0 - 4)}}{{(3 - 2) \cdot (3 - 4)}} +
\frac{{14 \cdot (0 - 2) \cdot (0 - 3)}}{{(4 - 2) \cdot (4 - 3)}}
\]
After simplifying the equation, the hidden value is determined to be 42.

\section{Distributed Key Generation}
The Distributed Key Generation (DKG) protocol is a cryptographic mechanism that allows multiple parties to generate a shared secret key collaboratively without relying on a single trusted authority. It ensures that no one party has complete knowledge of the secret key, thereby enhancing security and decreasing the likelihood of a single point of failure. In a DKG protocol, the participating parties collaborate to generate and distribute portions of the secret key. Combining these shares mathematically yields the final confidential key. Protocol phases include key generation, distribution, verification, and reconstruction. The primary benefits of DKG protocols are their security and resilience. The protocol mitigates the risk of a single party compromising the confidential key by distributing the key generation process across multiple parties. Even if some parties are compromised, the final key will remain secure if a minimum number of trustworthy parties are involved. Protocols for Distributed Key Generation have applications in numerous disciplines, including secure multi-party computation, cryptographic key management, threshold cryptography, and secure communication protocols. They provide a robust mechanism for establishing shared secret keys in situations where there is little or no trust between participants.\\
The Pedersen DKG protocol, proposed by Torben Pedersen \cite{pedersen1991threshold} in 1991, was used in the thesis. The Pedersen DKG protocol utilizes polynomial interpolation techniques and cryptographic primitives to achieve secure and distributed key generation. It provides a robust mechanism for establishing shared secret keys without relying on a trusted central authority. The protocol involves several steps, including key generation, sharing, verification, and reconstruction.The Pedersen DKG protocol utilizes polynomial interpolation techniques and cryptographic primitives to achieve secure and distributed key generation. It provides a robust mechanism for establishing shared secret keys without relying on a trusted central authority. The protocol involves several steps, including key generation, sharing, verification, and reconstruction.

\section{Smart contract}
\subfile{background/smartcontract.tex}

\section{Executor for DApps}
An essential part of enabling the execution of transactions on the blockchain network is played by the executor of a DApp. The backend element tasked with connecting with the blockchain and carrying out transactions on behalf of users is referred to as the executor in the context of DApps. The user's private key or mnemonic being kept in the backend is one typical method for carrying out transactions in a DApp. The private key, which is used to sign transactions and verify identities, is represented by a series of phrases known as the mnemonic. The executor can access the mnemonic when necessary to sign transactions on behalf of the frontend by securely keeping it in the backend. The executors play a specific role in the thesis as the verifiers and assigners of a Perdesen DKG protocol round within the smart contract. The decentralized system's executors serve as dependable parties and are in charge of assuring the fairness and security of the round assignment procedure.

\section{Social login for Web3}
\subsection{Web3 and Dapps}
Web3 and Decentralized Applications (DApps) have emerged as critical components of the evolution of the internet and digital ecosystems. Web3 is the vision of a more decentralized and user-centric internet, in which individuals have greater control over their data, identity, and digital interactions. It is a collection of technologies, protocols, and frameworks designed to empower users, cultivate trust, and facilitate peer-to-peer interactions. DApps, on the other hand, are applications that are created on top of decentralized networks and typically utilize blockchain technology. These applications inherit Web3's fundamental principles, including decentralization, transparency, and user ownership. From financial services and governance platforms to gaming and social media applications, they provide a variety of features. 
The development and adoption of Web3 and DApps have been supported by a growing body of research and innovation. Several academic papers and technical publications have contributed to the advancement of these technologies. For instance, the paper by Wood et al. titled "Ethereum: A Secure Decentralized Generalized Transaction Ledger" provides a comprehensive overview of the Ethereum platform and its underlying principles \cite{Ethereum}.
Another significant contribution is the work by Swan, who explores the concept of "Token Economy" in his book "Token Economy: How the Web3 Reinvents Value Exchange." The book delves into the transformative potential of tokenization and its implications for various industries \cite{swan2018token}. Moreover, the paper by Buterin et al. titled "A Next-Generation Smart Contract and Decentralized Application Platform" introduces the Ethereum platform, highlighting its unique features and use cases \cite{buterin2014next}. This paper serves as a foundational reference for understanding the capabilities and potential of DApps built on Ethereum.
\subsection{Social login and OAuth}
Social login is a prevalent method of authentication that enables users to log in to websites and applications using their existing social media accounts. Users are no longer required to establish new accounts and remember additional login credentials. Instead, users can merely click on a social media button, such as "Sign in with Facebook" or "Sign in with Google," to authenticate themselves. Social registration is supported by the OAuth2 (Open Authorization 2.0) \cite{oauth2} protocol, which provides a secure and standardized authentication framework. When a user logs in with a social media account, the website or application sends them to the respective social media platform for authentication. The user is then presented with a consent interface that describes the data to which the website or application requests access. Once the user grants permission, the social media platform provides the website or application with an access token that can be used to retrieve user information and authenticate the user's identity. Using OAuth2 for social authentication has multiple advantages. It increases security by removing the need for websites and applications to store user credentials. Instead, the obligation for authentication falls on the shoulders of the most reputable social media platforms. Second, social login streamlines the user experience by allowing users to log in with a few clicks and avoid the inconvenience of creating new accounts. It also allows websites and applications to utilize the extensive user profile data available on social media platforms, including user names, profile pictures, and email addresses, for personalization and customization.
\subsection{Social login benefits DApps}
Web3's technical complexity is one of the most significant obstacles it presents to normal consumers. The underlying technologies, such as blockchain, cryptographic keys, and smart contracts, can be complex and challenging for non-technical individuals to comprehend. Comprehending and traversing these complexities can be an impediment to entry, impeding widespread adoption and utility. In addition, the decentralized nature of Web3 platforms may result in fragmented user experiences and inconsistent user interfaces, making it difficult for non-technical users to effectively navigate and interact with decentralized applications. Simplifying the user experience and enhancing the accessibility of Web3 technologies will be crucial for overcoming these obstacles and ensuring that regular users can reap the full benefits of Web3.\\
This thesis proposes a solution to improve the usability of Web2 \cite{oreilly2005web2} applications by integrating smart contracts, the Pedersen Distributed Key Generation (DKG) protocol, and Shamir secret sharing. The present study endeavors to propose a solution that seeks to mitigate the difficulties encountered by laypersons in comprehending and maneuvering the intricate technicalities of Web3. Additionally, the solution endeavors to guarantee the secure generation and distribution of cryptographic keys, as well as the safeguarding of data.








%%%%%%%%%%%%%%%%%%%%%%%%%%%%%%%%%%%

\end{document}
