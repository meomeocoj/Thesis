\documentclass[../Main.tex]{subfiles}
\begin{document}
The concluding chapter of this thesis serves as a summary of the principal findings and contributions of the research presented in the antecedent chapters. In addition, we discuss the limitations of our study and acknowledge any restrictions that may have affected our findings. Ahead of time, we outline potential future research avenues, spotlighting unexplored areas that can build upon the foundation established by this thesis.
\section{Conclusion}
\label{section:6.1}
In conclusion, the system being discussed is an innovative and efficient solution for addressing identity and authorization challenges in the Web3 domain. This system has the potential to revolutionize online authentication and access by incorporating advanced features and functionalities. This solution utilizes advanced technologies such as blockchain, social login, and user-controlled encryption keys. It also incorporates robust security measures like smart contracts, Shamir's secret sharing, and the Pedersen DKG protocol. These features provide a seamless and secure user experience, enabling individuals to confidently engage with the Web3 ecosystem.\\
\indent The system's user-friendly interface facilitates convenient navigation through the authentication and authorization processes, making it a notable strength. The incorporation of social login enables a smooth and recognizable authentication process, minimizing obstacles and improving user acceptance. In addition, user-controlled encryption keys offer individuals increased control and ownership of their private data, effectively addressing the escalating concerns surrounding data privacy and security.\\
\indent Moreover, the system's capacity to streamline and automate the authorization process greatly improves efficiency and convenience. Virtualization technology enables the establishment of a uniform and stable deployment environment, facilitating the configuration and scalability of applications as required. The utilization of Express.js as the backend and JSON-RPC for communication enhances the system's robustness and performance, facilitating seamless and dependable interactions among various components.\\
\indent The system's capacity to tackle the urgent issues of identity and authorization in the Web3 domain renders it a valuable asset for individuals, developers, and businesses. Decentralized applications are being increasingly adopted and utilized across various industries, leading to innovation and growth within the Web3 ecosystem. The Web3 landscape is constantly changing, and the system is prepared to adapt and provide secure and efficient solutions for identity and authorization in the decentralized digital world.\\
\section{Future work}
The system's beta version shows promise by offering support for Google Auth0 and web browsers. The development has promising prospects for the system, as there are plans to enhance its compatibility and extend its reach. The roadmap entails incorporating support for additional reputable authentication providers, including Facebook, Apple, Reddit, and others. The system seeks to expand its user base and provide users with a greater variety of authentication options by integrating with well-known authentication services. This integration allows for seamless identity verification.\\
\indent Additionally, there are plans to improve the accessibility of the system by expanding its compatibility to various platforms, such as mobile devices and iPads. This expansion will enhance user interaction by accommodating various devices, thereby fostering inclusivity and improving user experience. The adoption of mobile platforms is essential for accommodating the growing number of users accessing the Web3 ecosystem via smartphones and tablets.\\
\indent In the context of cross-device support, the system currently does not offer the capability to share the private key across multiple devices. However, this feature has been identified as a critical target for future development and enhancement of the social login device. The ability to seamlessly and securely share the private key across different devices is crucial to providing users with a consistent and flexible experience.



\label{section:6.2}
\end{document}
