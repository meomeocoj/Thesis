\documentclass[../Main.tex]{subfiles}
\begin{document}
In this introductory chapter, we embark on an enlightening journey that establishes the foundation for our research project. Here, we present a comprehensive overview, delving into the motivations behind our research, the contributions it aims to make, and the well-structured roadmap that guides us through the investigation of our research domain.
\section{Motivation}
\label{section:1.1}
Blockchain technology, as exemplified by Bitcoin \cite{Bitcoin} and Ethereum \cite{Ethereum} networks, has experienced significant growth and garnered widespread attention due to its unique characteristics and prospective benefits. Blockchain has revolutionized many industries, including finance, supply chain, healthcare, and more, by providing a decentralized, transparent, and immutable platform for record-keeping and value transmission. However, conventional web technologies and systems offer their own set of benefits and advantages. Bridging the gap between blockchain and conventional web technologies can unleash a wealth of opportunities and synergies, resulting in a more robust and adaptable digital ecosystem. \\\
\indent One of the key benefits of blockchain technology lies in its ability to provide trust and transparency. The Bitcoin network, for instance, enables peer-to-peer transactions without the need for intermediaries, fostering trust among participants and reducing transaction costs. Ethereum, on the other hand, extends blockchain capabilities by supporting programmable smart contracts, enabling decentralized applications (DApps) with a wide range of use cases. Meanwhile, traditional web technologies offer a well-established infrastructure, user-friendly interfaces, and extensive compatibility with existing systems. By combining the benefits of both blockchain networks like Bitcoin and Ethereum and traditional web technologies, we can create a powerful hybrid solution that leverages the transparency of blockchain while maintaining the usability and familiarity of the traditional web. By facilitating self-sovereign identities and data control, blockchain technology promotes decentralization and empowers individuals. Users can have ownership and control over their digital assets and personal data, decreasing their dependence on centralized entities. This paradigm shift is facilitated by Bitcoin's decentralized network architecture and Ethereum's decentralized application platform. Traditional web technologies, on the other hand, provide users with convenience and familiarity via centralized authentication systems, social logins, and widespread standards. As demonstrated by Bitcoin and Ethereum, integrating these features into the blockchain ecosystem can improve user experience, encourage adoption, and bridge the gap between conventional web users and blockchain applications. \\\
\indent The growth and benefits of blockchain technology, exemplified by networks like Bitcoin and Ethereum, combined with the advantages of traditional web technologies, highlight the importance of bridging the gap between the two. By leveraging the strengths of both systems, we can create a hybrid solution that harnesses the transparency, security, and decentralization of blockchain while maintaining the usability, compatibility, and familiarity of the traditional web. This convergence unlocks new possibilities, expands the reach of blockchain applications, and paves the way for a more interconnected and inclusive digital future. Consequently, the objective of this thesis, a social login solution for Dapps using SSS and verified by DKG, is to combine the advantages of blockchain technology and conventional web authentication.
\section{Contributions}
\label{section:1.2}
Due to the fact that this solution is a large undertaking involving the implementation of numerous modules by numerous individuals, it is evident that I did not design and construct the system alone and that other developers participated in its creation. In addition, I was responsible for devising and implementing the mechanisms for the executors to share secrets and generate private keys for end users. I implement the majority of the project's features, with the exception of Shamir's algorithm for sharing secrets and the Distributed Key Generation protocol. In addition, I designed and implemented the majority of the data structures contained in smart contracts and the decentralized storage called Eueno. In addition, this system has a unique architecture for securing and enriching the user experience, as well as enabling developers to integrate existing Dapps seamlessly.


\section{Thesis structure}
\label{section:1.3}
The present thesis is organized into six distinct chapters, each of which fulfills a specific objective in the comprehensive investigation of the research subject matter. Chapter 1 serves as the introductory section of this thesis, wherein the underlying motivation driving the study is established. Furthermore, this chapter highlights the significant contributions made by the research and provides a comprehensive outline of the overall structure of the thesis. Chapter 2 of this thesis aims to establish a comprehensive understanding of fundamental concepts that are crucial to the subject matter. These concepts include blockchain, transactions, blocks, wallets, Shamir's secret sharing, distributed key generation, smart contracts, executor for decentralized applications (DApps), and social login for Web3. By delving into these concepts, this chapter lays the groundwork for the subsequent analysis and exploration of the topic at hand. Chapter 3 of this study presents the proposed solution, which outlines an innovative approach that has been adopted to effectively tackle the challenges that have been identified. Chapter 4 delves into an in-depth analysis of the technical issues and design considerations encountered throughout the implementation process. This chapter aims to address and shed light on the various challenges that were confronted during the execution of the project. Chapter 5 of this study presents a comprehensive evaluation of the proposed solution, offering valuable insights into its performance and usability. In conclusion, Chapter 6 serves as the final segment of this thesis, wherein the findings are succinctly summarized and potential avenues for future research are deliberated upon. The inclusion of a reference section within a thesis serves the purpose of meticulously documenting all the sources that have been cited throughout the research, thereby upholding the academic integrity of the study.
\end{document}
