\documentclass[../../Main.tex]{subfiles}
\begin{document}
\subsection{History and defination}
Smart contracts are agreements that automatically carry out their obligations because they are encoded in code. By eliminating the need for middlemen and supplying a safe and decentralized method to facilitate and enforce agreements or transactions, these contracts automatically execute and enforce themselves. The idea of smart contracts has been around since the 1990s, and computer scientist Nick Szabo \cite{szabo1997formalizing} is credited with coining the term. However, smart contracts did not receive much attention or widespread use until the advent of blockchain technology, particularly with the launch of Ethereum \cite{Ethereum} in 2015. A Turing-complete programming language was introduced by Ethereum, a decentralized blockchain platform, allowing for the creation and execution of sophisticated smart contracts. This innovation paved the way for the development of decentralized applications (DApps) that might use smart contracts to secure and automate a variety of activities, including voting systems, supply chain management, and financial transactions. Since then, smart contracts have become more well-known and are being investigated in a variety of sectors and industries for their potential to transform conventional corporate operations. Their immutability and transparency, along with the ability to automate processes and get rid of middlemen, have the potential to improve workflow, boost productivity, and cut costs.
\subsection{Practical use cases}
Smart contracts have become a game-changing technology with several real-world applications in a wide range of industries. These self-executing contracts, which are inscribed on a blockchain, allow for secure and automated transactions, doing away with the need for middlemen and enhancing participant trust. Smart contracts have transformed lending platforms, decentralized exchanges, and yield farming protocols in the field of DeFi \cite{vanderveen2021defi}. Smart contracts offer a transparent and effective ecosystem for decentralized financial applications by automating financial transactions and following established regulations. Likewise, supply chain management has benefited from smart contracts. By facilitating seamless tracking and verification of commodities along the supply chain, these contracts improve traceability, lower fraud, and streamline logistical operations. Smart contracts have simplified real estate transactions by enabling property transfers, escrow services, and rental agreements. Smart contracts increase efficiency and transparency in the real estate sector by doing away with middlemen and automating repetitive operations. \\
\indent In this thesis, the smart contract plays a pivotal role in the ecosystem by fulfilling a multitude of significant responsibilities. The storage and maintenance of configuration updates for the Perdesen Distributed Key Generation (DKG) protocol is a primary responsibility. The smart contract is responsible for managing a whitelist of decentralized applications (DApps) that utilize the aforementioned solution. This mechanism ensures that only authorized DApps are able to participate in the protocol. The smart contract is designed to enable the asynchronous execution of the Perdesen DKG protocol rounds, which is a fundamental feature of its functionality. The process entails the antecedent creation of cryptographic keys through the secure retention of encrypted shares for each node involved in the operation.  The implementation of a smart contract facilitates the asynchronous execution of the key generation procedure, thereby enhancing operational efficiency and scalability. The transparent management of the key generation process is regarded as a fundamental characteristic of the smart contract. The implementation of transparency in the process is paramount to guaranteeing the privacy and security of participants' private keys. The provision of transparency enables participants to authenticate the advancement and soundness of the key generation procedure while upholding the confidentiality of their private key data. The smart contract assumes a crucial function in the verification of signatures from nodes that are involved in the process. The implementation of a verification process within the Perdesen Distributed Key Generation (DKG) protocol serves to guarantee that exclusively legitimate users are allocated roles during each round of the cryptographic scheme. Through the process of signature verification, the smart contract ensures the integrity and authenticity of the nodes involved, thereby augmenting the overall security and dependability of the protocol.

\end{document}
