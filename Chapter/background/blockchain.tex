\documentclass[../../Main.tex]{subfiles}
\begin{document}
\subsection{Transactions}
\label{section:2.1.1}
"Transactions are the most important part of the Bitcoin system. Everything else in bitcoin is designed to ensure that transactions can be created, propagated on the network, validated, and finally added to the global ledger of transactions (the blockchain). Transactions are data structures that encode the transfer of value between participants in the Bitcoin system. Each transaction is a public entry in bitcoin’s blockchain, the global double-entry bookkeeping ledger" according to "Mastering Bitcoin: Unlocking Digital Cryptocurrencies" by Andreas M. Antonopoulos \cite{antonopoulos2014masteringbitcoin}, which implies that transactions are fundamental components of blockchain technology, serving as the building blocks for the transfer and exchange of digital assets. Blockchain networks' security and trustworthiness rely heavily on transactions. They are intended to be verifiable and immutable, providing a transparent and auditable log of all blockchain activities. By recording each transaction on the distributed ledger, participants are able to trace the history and origin of digital assets, fostering accountability and preventing double spending. Multiple stages are involved in the creation of a transaction. The sender initiates the transaction by specifying the recipient's address and the desired transfer amount. The originator then signs the transaction with their private key, ensuring the transaction's authenticity and integrity. Once the transaction has been digitally signed, it is disseminated to the network for validation and inclusion in a block. The validation procedure involves verifying the digital signature of the transaction using the sender's public key, thereby ensuring that the transaction has not been tampered with and that the originator has sufficient funds to complete the transfer. The transaction is submitted to a pool of pending transactions awaiting confirmation after validation. Miners, who are tasked with safeguarding the blockchain, select transactions from the pool and incorporate them into a new block. A consensus mechanism, such as proof-of-work or proof-of-stake, is then used to add the transaction to the blockchain.The creation of transactions on the blockchain enables participants to transmit digital assets without the need for intermediaries in a transparent and secure manner. It assures the system's integrity by employing cryptographic techniques to authenticate and authorize transactions, thereby rendering the process tamper-proof and fraud-resistant.
\subsection{Blocks}
\label{section:2.1.2}
A block in a blockchain is a fundamental element that is crucial to the network's structure and functionality. It functions as a repository for a collection of transactions and other pertinent data. Each block is comprised of a block preamble, which includes metadata such as the block's unique identifier, timestamp, and a reference to the previous block, establishing a chronological order. The block contains transactions, which represent numerous actions within the blockchain network. These transactions include sender and recipient addresses, digital signatures for authentication, and additional pertinent information. The block also contains a Merkle tree root \cite{merkle1987digital}, which provides an efficient method for verifying the validity of transactions contained within the block. In addition, each block is allocated a unique block hash that is generated by a cryptographic hash function. This block hash serves as a digital fingerprint for the block's content and ensures its immutability. In proof-of-work consensus algorithms, for instance, miners compete to find a nonce value that, when combined with the block header, satisfies specific criteria, thereby adding a layer of security through the solution of computational puzzles. The block functions as the fundamental unit of the blockchain, enabling secure, transparent, and efficient transaction storage and verification.
\subsection{Wallet}
\label{section:2.1.3}
\subsubsection{Key and address}
\label{section:2.1.3.1}
Key and address are foundational concepts pertaining to user identification and transaction security in the context of blockchain technology and cryptocurrencies. A key, also known as a cryptographic key, is a fragment of information utilized in cryptographic algorithms for a variety of purposes, including encryption, decryption, and digital signatures. Typically, in the context of blockchain, keys are used to secure access to digital assets and to authenticate transactions. There are various mathematically related key categories, including private and public keys. A private key is a secret, randomly generated number that is kept covert by the user. It is used to generate digital signatures, which verify the integrity and authenticity of transactions. The private key should be stored in a secure location and never shared with anyone. If a third party obtains access to the private key, they may be able to take control of the associated digital assets. On the other hand, an address is a cryptographic representation of a user's public key. In a blockchain network, it is a string of alphanumeric characters that functions as a unique identifier for receiving transactions or messages. The public key is used to generate addresses, but they do not disclose any information about the private key. When sending a transaction to a particular user in a blockchain network, the recipient's address is used as the destination. The address functions as a pseudonymous identifier, providing privacy and security. The recipient can then access and manage the digital assets associated with that address using their private key.
\subsubsection{Wallet}
\label{section:2.1.3.2}
A cryptocurrency wallet is a software application or hardware device that enables users to store, administer, and interact with their digital assets in a secure manner. Wallets play a crucial role in the adoption and use of cryptocurrencies by both consumers, enhancing the overall experience with a variety of advantages. Wallets provide a convenient and intuitive interface for managing digital assets. They provide a secure solution for storing private keys, which are required for accessing and controlling cryptocurrencies. Wallets enable users to transfer and receive funds, track their transaction history, and monitor their account balances by storing private keys securely. Wallets typically include features such as address book administration, transaction history, and real-time market data, providing users with a comprehensive set of tools for managing their cryptocurrency holdings.
One important aspect of wallet security is the implementation of industry standards, such as the BIP39 specification \cite{bip39}, in the generation and management of mnemonic phrases or seed phrases. The BIP39 specification ensures that wallets adhere to a standardized method for generating mnemonic phrases, which are human-readable sets of words. These phrases can be used to derive the cryptographic keys necessary to access and manage cryptocurrency funds. By adopting the BIP39 specification, wallets provide users with a consistent and reliable way to backup and restore their wallets, offering an additional layer of security and ease of use.
MetaMask \cite{metamask} is a prominent example of a cryptocurrency wallet. MetaMask is a wallet extension for web browsers that enables users to interact with Ethereum-based decentralized applications (DApps) directly from the browser. It provides a user-friendly and secure interface for interacting with Ethereum accounts and the blockchain. MetaMask provides a straightforward and intuitive interface that integrates seamlessly with popular web browsers such as Chrome, Firefox, and Brave. Within minutes, users can install the MetaMask extension and configure their Ethereum wallet. After configuring the wallet, users can access their Ethereum accounts, view their token balances, and conduct transactions.
\subsection{Consensus}
The concept of consensus holds paramount importance in the realm of blockchain technology as it serves to guarantee the agreement and validity of transactions throughout the network. This mechanism encompasses a process by which decentralized nodes within the network collectively establish agreement regarding the current state of the blockchain. This paper examines two prevalent consensus algorithms, namely Proof of Work (PoW) and Proof of Stake (PoS).\\
\indent PoW consensus algorithm, initially pioneered by Bitcoin \cite{Bitcoin}, serves as the foundational mechanism for validating transactions and maintaining the integrity of the blockchain network. In the context of PoW, the participants referred to as miners engage in a competitive process aimed at solving intricate mathematical puzzles. In the realm of blockchain technology, the initial miner who successfully unravels the intricate puzzle is duly acknowledged and bestowed with a reward, subsequently appending a novel block to the existing chain of transactions. The aforementioned process necessitates a substantial amount of computational resources and incurs a considerable level of energy expenditure. The security of a blockchain system is upheld through the utilization of PoW, which effectively deters malicious entities from tampering with previous transactions. This is achieved by imposing a significant computational burden on any attempts to modify the blockchain's historical records.\\
\indent PoS \cite{Ethereum} consensus algorithm serves as a viable alternative to the PoW mechanism, with the primary objective of mitigating the concerns pertaining to energy consumption commonly associated with PoW. In the PoS consensus mechanism, the selection of validators to generate new blocks is determined by their cryptocurrency holdings and their willingness to "stake" said holdings as collateral. The selection of validators is typically conducted through a deterministic procedure, which frequently takes into consideration factors such as the magnitude of their stake and the duration for which they have maintained it. PoS consensus mechanism is widely acknowledged for its superior energy efficiency in comparison to the PoW mechanism, primarily due to its reduced reliance on intensive computational resources.\\
\indent The utilization of both PoW and PoS consensus mechanisms presents distinct benefits and limitations. PoW consensus mechanism is renowned for its robust security measures, albeit at the cost of significant resource consumption. Conversely, the PoS protocol boasts energy efficiency advantages, yet it may exhibit vulnerability to specific forms of attacks. The selection between PoW and PoS is contingent upon the distinct objectives and prerequisites of a blockchain network.\\


\end{document}
