\documentclass[a4paper,13pt,3p,twoside]{report}
\usepackage{scrextend}
\usepackage[utf8]{vietnam}
\usepackage[top=2cm, bottom=2cm, left=3.5cm, right=2.5cm]{geometry}
\usepackage{graphicx} % Cho phép chèn hỉnh ảnh
\usepackage{fancybox} % Tạo khung box
\usepackage{indentfirst} % Thụt đầu dòng ở dòng đầu tiên trong đoạn
\usepackage{amsthm} % Cho phép thêm các môi trường định nghĩa
\usepackage{latexsym} % Các kí hiệu toán học
\usepackage{amsmath} % Hỗ trợ một số biểu thức toán học
\usepackage{amssymb} % Bổ sung thêm kí hiệu về toán học
\usepackage{amsbsy} % Hỗ trợ các kí hiệu in đậm
\usepackage{times} % Chọn font Time New Romans
\usepackage{array} % Tạo bảng array
\usepackage{enumitem} % Cho phép thay đổi kí hiệu của list
\usepackage{subfiles} % Chèn các file nhỏ, giúp chia các chapter ra nhiều file hơn
\usepackage{titlesec} % Giúp chỉnh sửa các tiêu đề, đề mục như chương, phần,..
\usepackage{titletoc}
\usepackage{chngcntr} % Dùng để thiết lập lại cách đánh số caption,..
\usepackage{pdflscape} % Đưa các bảng có kích thước đặt theo chiều ngang giấy
\usepackage{afterpage}
\usepackage[ruled,vlined]{algorithm2e}  % Hỗ trợ viết các giải thuật
\usepackage{capt-of} % Cho phép sử dụng caption lớn đối với landscape page
\usepackage{multirow} % Merge cells
\usepackage{fancyhdr} % Cho phép tùy biến header và footer
\usepackage[natbib,backend=biber,style=ieee]{biblatex} % Giúp chèn tài liệu tham khảo
\usepackage{appendix}
\usepackage{outlines}
\usepackage[font=small,labelfont=bf]{caption}

\usepackage{listings}
\usepackage{float}
\usepackage{subcaption}
\usepackage{xurl}

\usepackage[nonumberlist, nopostdot, nogroupskip, acronym]{glossaries}
\usepackage{glossary-superragged}

\newglossarystyle{myGlossaryStyle}{%
  \renewenvironment{theglossary}{%
    \begin{description}[align=left,labelwidth=3cm]%
  }{%
    \end{description}%
  }%
  \renewcommand*{\glossaryentryfield}[5]{%
    \item[\glstarget{##1}{##2}] ##3%
  }%
  \renewcommand*{\glossarysubentryfield}[6]{%
    \item[\glstarget{##2}{##3}] ##4%
  }%
}
\setglossarystyle{myGlossaryStyle}
\usepackage{setspace}
\usepackage{parskip}

% package content table
\usepackage{tocbasic}

\usepackage{blindtext}


% ===================================================

\renewcommand{\bibname}{reference} 

\renewcommand{\figurename}{Figure}
\renewcommand{\tablename}{Table}
\renewcommand{\chaptername}{CHAPTER}

\addbibresource{reference.bib} % chèn file chứa danh mục tài liệu tham khảo vào 

\include{lstlisting} % Phần này cho phép chèn code và formatting code như C, C++, Python

%\makeglossaries
\makenoidxglossaries

% Danh mục thuật ngữ và từ viết tắt
\newglossaryentry{iaas}{
    type=\acronymtype,
    name={IaaS},
    description={Infrastructure as a Service},
}

\newglossaryentry{SSS}{
    type=\acronymtype,
    name={SSS},
    description={Shamir's secret sharing},
}

\newglossaryentry{DKG}{
    type=\acronymtype,
    name={DKG},
    description={Distributed Key Generation},
}


\newglossaryentry{HTTP}{
    type=\acronymtype,
    name={HTTP},
    description={Hyper Text Transfer Protocol}
}

\newglossaryentry{Dapp}{
    type=\acronymtype,
    name={Dapp},
    description={Decentralize application}
}

\newglossaryentry{BIP}{
    type=\acronymtype,
    name={BIP},
    description={Bitcoin Improvement Proposal}
}

\newglossaryentry{OAuth2}{
    type=\acronymtype,
    name={OAuth},
    description={Open Authenticate 2.0}
}
\newglossaryentry{DeFi}{
    type=\acronymtype,
    name={DeFi},
    description={Decentralize finance}
}

\newglossaryentry{PoS}{
    type=\acronymtype,
    name={PoS},
    description={Proof of Stake}
}

\newglossaryentry{PoW}{
    type=\acronymtype,
    name={PoW},
    description={Proof of Work}
}


% ===================================================


\fancypagestyle{plain}{%
\fancyhf{} % clear all header and footer fields
\fancyfoot[RO,RE]{\thepage} %RO=right odd, RE=right even
\renewcommand{\headrulewidth}{0pt}
\renewcommand{\footrulewidth}{0pt}}

\setlength{\headheight}{10pt}

\def \TITLE{A social login solution for Web3 using Shamir's secret sharing and verified DKG}
\def \AUTHOR{NGUYEN TUAN MINH}

% ===================================================
\titleformat{\chapter}[hang]{\centering\bfseries}{CHƯƠNG \thechapter.\ }{0pt}{}[]

\titleformat 
    {\chapter} % command
    [hang] % shape
    {\centering\bfseries} % format
    {CHAPTER \thechapter.\ } % label
    {0pt} %sep
    {} % before
    [] % after
\titlespacing*{\chapter}{0pt}{-20pt}{20pt}

\titleformat
    {\section} % command
    [hang] % shape
    {\bfseries} % format
    {\thechapter.\arabic{section}\ \ \ \ } % label
    {0pt} %sep
    {} % before
    [] % after
\titlespacing{\section}{0pt}{\parskip}{0.5\parskip}

\titleformat
    {\subsection} % command
    [hang] % shape
    {\bfseries} % format
    {\thechapter.\arabic{section}.\arabic{subsection}\ \ \ \ } % label
    {0pt} %sep
    {} % before
    [] % after
\titlespacing{\subsection}{30pt}{\parskip}{0.5\parskip}

\titleformat
    {\subsubsection} % command
    [hang] % shape
    {\itshape} % format
    {\thechapter.\arabic{section}.\arabic{subsection}.\arabic{subsubsection}\ \ \ \ } % label
    {0pt} %sep
    {} % before
    [] % after
\titlespacing{\subsubsection}{50pt}{\parskip}{0.5\parskip}




% ===================================================
\usepackage{hyperref}
\hypersetup{pdfborder = {0 0 0}} %
\hypersetup{pdftitle={\TITLE},
	pdfauthor={\AUTHOR}}
	
\usepackage[all]{hypcap} % Cho phép tham chiếu chính xác đến hình ảnh và bảng biểu

\graphicspath{{figures/}{../figures/}} % Thư mục chứa các hình ảnh

\counterwithin{figure}{chapter} % Đánh số hình ảnh kèm theo chapter. Ví dụ: Hình 1.1, 1.2,..

\title{\bf \TITLE}
\author{\AUTHOR}

\setcounter{secnumdepth}{3} % Cho phép subsubsection trong report
% \setcounter{tocdepth}{3} % Chèn subsubsection vào bảng mục lục

\theoremstyle{definition}
\newtheorem{example}{Ví dụ}[chapter] % Định nghĩa môi trường ví dụ

\onehalfspacing
%Khoảng cách xuống dòng
\setlength{\parskip}{6pt}
%Lùi đầu dòng
\setlength{\parindent}{15pt}



% =========================== BODY ===============
\begin{document}
% \newgeometry{top=2cm, bottom=2cm, left=2cm, right=2cm}
\subfile{cover} % Phần bìa
% \restoregeometry

% ===================================================
\pagenumbering{roman}

\renewcommand{\figurename}{Figure}
\renewcommand{\tablename}{Table}
\renewcommand{\chaptername}{CHAPTER}
% \pagestyle{empty} % Header và footer rỗng
%\newpage
%\subfile{chapters/0_1_subject.tex}

\newpage 
\subfile{Chapter/0_1_requirement_for_the_thesis.tex}

\newpage
\subfile{Chapter/0_2_acknowledgment.tex}

%\newpage
%\subfile{Chapter/0_3_Tom_tat_noi_dung.tex}

\newpage
\subfile{Chapter/0_3_abstract.tex}

% ===================================================
% \pagestyle{empty} % Header và footer rỗng


\titlecontents{chapter}
    [0.0cm]             % left margin
    {\bfseries\vspace{0.3cm}}                  % above code
    {{\bfseries{\scshape} CHAPTER \thecontentslabel.\ }}
    % numbered format
    {}         % unnumbered format
    {\titlerule*[0.3pc]{.}\contentspage}         % filler-page-format, e.g dots

\titlecontents{section}
    [0.0cm]             % left margin
    {\vspace{0.3cm}}                  % above code
    {\thecontentslabel \ } % numbered format
    {}         % unnumbered format
    {\titlerule*[0.3pc]{.}\contentspage}         % filler-page-format, e.g dots
    
\titlecontents{subsection}
    [1.0cm]             % left margin
    {\vspace{0.3cm}}                  % above code
    {\thecontentslabel \ } % numbered format
    {}         % unnumbered format
    {\titlerule*[0.3pc]{.}\contentspage}         % filler-page-format, e.g dots

 % Tạo mục lục tự động
\addtocontents{toc}{\protect\thispagestyle{empty}}


\pagestyle{empty}
\renewcommand*\contentsname{TABLE OF CONTENTS}
\tableofcontents 
\clearpage

% \pagenumbering{roman}
%Tạo danh mục hình vẽ.
\renewcommand{\listfigurename}{LIST OF FIGURES}
{\let\oldnumberline\numberline
\renewcommand{\numberline}{Figure~\oldnumberline}
\listoffigures} 
% \phantomsection\addcontentsline{toc}{section}{\numberline {} DANH MỤC HÌNH VẼ}
\newpage

 %Tạo danh mục bảng biểu.
\renewcommand{\listtablename}{LIST OF TABLES}
{\let\oldnumberline\numberline
\renewcommand{\numberline}{Bảng~\oldnumberline}
\listoftables}
% \phantomsection\addcontentsline{toc}{section}{\numberline {} DANH MỤC BẢNG BIỂU}
\glsaddall 
% \renewcommand*{\glossaryname}{Danh sách thuật ngữ}


\renewcommand*{\acronymname}{ACRONYMS}
\printnoidxglossaries
% ===================================================


\newpage
\pagenumbering{arabic}

\pagestyle{fancy}
\fancyhf{}
\fancyhead[RE, LO]{\leftmark}
%\fancyhead[LE]{\rightmark}
\fancyfoot[RE, LO]{\thepage}

\chapter{INTRODUCTION}
\label{chapter:Introduction}
\subfile{Chapter/1_Introduction} % Phần mở đầu

\newpage
%\pagestyle{fancy} % Áp dụng header và footer
\chapter{BACKGROUND}
\label{chapter:Background}
\subfile{Chapter/2_Background}

\newpage
%\pagestyle{fancy} % Áp dụng header và footer
\chapter{SOLUTION}
\label{chapter:Solution}
\subfile{Chapter/3_Solution}

\newpage
%\pagestyle{fancy} % Áp dụng header và footer
\chapter{TECHNICAL ISSUES AND DESIGN}
\label{chapter:Technical_issues_and_design}
\subfile{Chapter/4_Technical_issues_and_design}

\newpage
%\pagestyle{fancy} % Áp dụng header và footer
\chapter{EVALUATION}
\label{chapter:Evaluation}
\subfile{Chapter/5_Evaluation}
\newpage
%\pagestyle{fancy} % Áp dụng header và footer
\chapter{CONCLUSION AND FUTURE WORK} %Kết luận và hướng phát triển}
\label{chapter:conclusion}
\subfile{Chapter/6_Conclusion}

\newpage
\renewcommand\bibname{REFERENCE}
\printbibliography
\phantomsection\addcontentsline{toc}{chapter}{REFERENCE}

\end{document}
